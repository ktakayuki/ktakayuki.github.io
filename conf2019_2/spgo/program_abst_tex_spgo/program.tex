\documentclass[a4]{article}
\setlength{\topmargin}{-1.3cm}
\addtolength{\textheight}{4cm}
\setlength{\oddsidemargin}{-0.4truemm}  
\setlength{\evensidemargin}{-0.4truemm} 
\addtolength{\textwidth}{40truemm} 
\newcommand{\todayd}{\the\year/\the\month/\the\day}

\usepackage{amssymb}
\usepackage{amsmath}
\usepackage{amsthm}
\usepackage{amscd}
\usepackage{graphicx}
%\usepackage[dvips]{color}
\usepackage[dvipdfm, usenames]{color}
\usepackage[all]{xy}
\usepackage{comment}
\usepackage{setspace}
\usepackage{utf}

\newcommand{\mathsym}[1]{{}}
\newcommand{\unicode}[1]{{}}

\newcounter{mathematicapage}


%%%%%%%%% Theorem-like environment %%%%%%%%%%%
%
\theoremstyle{plain} %text of this environment is typesetted in italics
\newtheorem{theorem}{\indent\sc Theorem}[section]
\newtheorem{lemma}[theorem]{\indent\sc Lemma}
\newtheorem{corollary}[theorem]{\indent\sc Corollary}
\newtheorem{proposition}[theorem]{\indent\sc Proposition}
\newtheorem{claim}[theorem]{\indent\sc Claim}
\newtheorem{conjecture}[theorem]{\indent\sc Conjecture}
%
\theoremstyle{definition} %text of this environment is typesetted in roman letters
\newtheorem{definition}[theorem]{\indent\sc Definition}
\newtheorem{remark}[theorem]{\indent\sc Remark}
\newtheorem{example}[theorem]{\indent\sc Example}
\newtheorem{notation}[theorem]{\indent\sc Notation}
\newtheorem{assertion}[theorem]{\indent\sc Assertion}
\newtheorem{observation}[theorem]{\indent\sc Observation}
\newtheorem{problem}[theorem]{\indent\sc Problem}
\newtheorem{question}[theorem]{\indent\sc Question}
%
%If a theorem-like environment should not be numbered,
%add * after \newtheorem, and delete the counter option such as [theorem].
\newtheorem*{remark0}{\indent\sc Remark}
%
%%%%% Proof %%%%%
\renewcommand{\proofname}{\indent\sc Proof.}
%The following commands are available in the proof environment:
%\begin{proof}
%\end{proof}
%The end of a proof is marked with a square.
%%%%%%%%%%%%%%%%%%%%%%%%%%%%%%%%%%%%%%%%%

\begin{document}

\begin{center}
  {\huge Studies on Pseudoconvexity of General Order}
\vskip5mm
{\large 大阪市立大学, 2020/3/8-10}\footnote{This conference is supported by Osaka City University Advanced Mathematical Institute: MEXT Joint Usage/Research Center on Mathematics and Theoretical Physics (共同利用・共同研究, 一般, C, "一般位数の擬凸性の研究"). }
\end{center}

\vskip8mm
\noindent{\Large \bf Program}
\vskip5mm
\noindent{\bf March 8}
\vskip2mm
\noindent {\bf 9:50 -- 10:50 Masanori Adachi (Shizuoka)}\\
The Diederich--Fornaess index and its variants
\vskip3mm
\noindent {\bf 11:00 -- 11:50  Makoto Abe (Hiroshima)}\\
A Characterization of Subpluriharmonicity for a Function of Several Complex Variables
\vskip3mm
\noindent {\bf 13:30--15:00 Thomas Pawlaschyk (Wuppertal)}\\
On $q$-plurisubharmonic functions in $\mathbb{C}^n$
\vskip3mm
\noindent {\bf 15:30 -- 16:30 Jihun Yum (Busan)}\\
Diederich--Fornaess index and Steinness index for abstract CR-manifolds
\vskip8mm
\noindent{\bf March 9}
\vskip2mm
\noindent {\bf 10:00 -- 11:30 Thomas Pawlaschyk (Wuppertal)}\\
On $q$-pseudoconvex domains in $\mathbb{C}^n$
\vskip3mm
\noindent {\bf 13:30 -- 14:30 Makoto Abe (Hiroshima)}\\
Intermediate pseudoconvexity for unramified Riemann domains over $\mathbb{C}^n$
\vskip3mm
\noindent {\bf 14:40 -- 15:40 Kazuko Matsumoto (Tokyo University of Science)}\\
On the theorems of Rothstein and Sperling concerning continuations of analytic sets
\vskip3mm
\noindent {\bf 15:50 -- 16:50 Shun Sugiyama (Hiroshima)}\\
A new proof of  theorem of Eastwood--Vigna--Suria
\vskip8mm
\noindent{\bf March 10}
\vskip2mm
\noindent {\bf 9:00 -- 10:30 Thomas Pawlaschyk (Wuppertal)}\\
On $q$-pseudoconcave sets in $\mathbb{C}^n$
\vskip3mm
\noindent {\bf 10:40 -- 11:30 Takeo Ohsawa (Nagoya)}\\
Extension problems and notions of convexity


\newpage


\noindent{\Large \bf Abstracts}
\vskip5mm
\noindent {\bf Masanori Adachi (Shizuoka)}\\
The Diederich--Fornaess index and its variants
\vskip3mm
The Diederich--Fornaess index is a classical notion to measure
the strength of pseudoconvexity for weakly pseudoconvex domains
and known to have applications in the $L^2$ (Sobolev) esimates for the dbar operator.
This talk shall give an overview on recent developments around this index
including its estimates (by Levi rank, D'Angelo 1-form, or other geometric conditions)
and its variants (for pseudoconcavity, rough boundary, or CR manifolds).
In particular, we shall illustrate some of the results for the complements of Levi-flats.
\vskip5mm
\noindent {\bf Makoto Abe (Hiroshima)}\\
A Characterization of Subpluriharmonicity for a Function of Several Complex Variables
\vskip3mm
We give a characterization of a subpluriharmonic function of several complex variables
in the sense of Fujita (J. Math. Kyoto Univ., {\bf 30}: 637--649, 1990) by using polynomial
functions of degree at most two.  This is joint work with Shun Sugiyama. 
\vskip5mm
\noindent {\bf Thomas Pawlaschyk (Wuppertal)}\\
On $q$-plurisubharmonic functions in $\mathbb{C}^n$ (March 8)
\vskip3mm
I will talk about upper semi-continuous $q$-plurisubharmonic functions in the sense of Hunt-Murray and compare with other notions of generalized convex functions. I will present different characterizations of $q$-plurisubharmonic functions and show approximation techniques. 
\vskip5mm
\noindent {\bf Jihun Yum (Busan)}\\
Diederich--Fornaess index and Steinness index for abstract CR-manifolds
\vskip3mm
Let $\Omega$ be a smooth bounded pseudoconvex domain in $\mathbb{C}^n$. 
The Diederich--Fornaess index and the Steinness index of $\Omega$ are defined by 
\[
DF(\Omega) := \sup_\rho \{0 < \eta < 1 : -(-\rho)^\eta\ \text{is}\ \text{strictly}\ \text{plurisubharmonic}\ \text{on}\ \Omega\},
\]
\[
S(\Omega) := \inf_\rho\{\eta > 1 : \rho^\eta\ \text{is}\ \text{strictly}\ \text{plurisubharmonic}\ \text{on}\ \overline{\Omega}^{\complement}\cap U\ \text{for}\ \text{some}\ \text{neighborhood}\ U\ \text{of}\ \partial\Omega\}.
\]
Roughly speaking, $DF(\Omega)$ is the supremum of the H\"older exponents of these exhaustions near the boundary, and $S(\Omega)$ is the infimum of the H\"older exponents of positive strictly plurisubharmonic functions in $\overline\Omega^{\complement}$ which approaches to zero on the boundary $\partial\Omega$. We first describe two indices in terms of a special $1$-form on $\partial\Omega$, called the D'Angelo $1$-form. Then we will see how this characterization allows us to define two indices for abstract CR-manifolds of hypersurface type. 
\vskip5mm
\noindent {\bf Thomas Pawlaschyk (Wuppertal)}\\
On $q$-pseudoconvex domains in $\mathbb{C}^n$  (March 9)
\vskip3mm
I will present a list of different characterizations of $q$-pseudoconvex domains which can be defined by, e.g., $q$-plurisubharmonic exhaustion functions. Furthermore, I will give examples of $q$-pseudoconvex domains. 
\vskip5mm
\noindent {\bf Makoto Abe (Hiroshima)}\\
Intermediate pseudoconvexity for unramified Riemann domains over $\mathbb{C}^n$
\vskip3mm
We characterize the $q$-pseudoconvexity for unramified Riemann domains over $\mathbb{C}^n$,
where $1 \leq q \leq n$, by the continuity property which holds for a class of maps whose
projections to $\mathbb{C}^n$ are families of unidirectionally parameterized $q$-dimensional
analytic balls written by polynomials of degree at most two. 
This is joint work with Tadashi Shima and Shun Sugiyama.
\vskip5mm
\noindent {\bf Kazuko Matsumoto (Tokyo University of Science)}\\
On the theorems of Rothstein and Sperling concerning continuations of analytic sets
\vskip3mm
We generalize some theorems of Rothstein--Sperling (1966) concerning continuations of analytic sets by using of the notions of pseudoconvex domains of general order (or equivalently, locally $q$-complete domains with corners in the sense of Diederich--Fornaess). 

The results are as follows. \\
(1) Let $D$ be a bounded domain of pseudoconvex of order $k$ $(0 \leq k \leq n-1)$ in $\mathbb{C}^n$ $(n \geq 2)$ and let $S$ be an irreducible analytic set of dimension
$m$ in a neighborhood of the boundary $\partial D$. If $m \geq n - k + 1$, then $S$ is
continuable to an analytic set in a neighborhood of $\overline{D}$.\\
(2) In addition to (1), suppose that the domain $D$ is defined by $D :=
\{ z\in \mathbb{C}^n : \varphi(z) < 0\}$ for some $\varphi\colon \mathbb{C}^n \to \mathbb{R}$ which is pseudoconvex exhaustion function of order $k$. 
If $m \geq \max (n-k, 2)$, then $S$ is continuable to an analytic set in a neighborhood of $\overline{D}$. \\
(3) The bounds of dimension $m$ of S in (1) and (2) are best.\\
This is a joint work with O. Fujita (1931-2018). 
\noindent The three examples in (3) are due to him only.
\vskip5mm
\noindent {\bf Shun Sugiyama (Hiroshima)}\\
A new proof of  theorem of Eastwood--Vigna--Suria
\vskip3mm
In this talk, I will explain the relation between $q$-pseudoconvexity and an algebraic condition of cohomology groups. Especially, I will introduce a new proof of theorem of Eastwood--Vigna--Suria. This proof is based on Kajiwara-Kazama's method in 1973.
\vskip5mm
\noindent {\bf Thomas Pawlaschyk (Wuppertal)}\\
On $q$-pseudoconcave sets in $\mathbb{C}^n$  (March 10)
\vskip3mm
The $q$-pseudoconcave sets are closed sets in $\mathbb{C}^n$ whose complement is $q$-pseudoconvex. They occur naturally as analytic sets of certain minimal dimension. On the converse, one may ask whether a closed $q$-pseudoconcave set $A$ admits a local foliation by complex submanifolds. I will give a positive answer in the case when $A$ is the graph of a continuous mapping $f\colon \mathbb{C}^n\times\mathbb{R}\to\mathbb{R}\times \mathbb{C}^p$. 
\vskip5mm
\noindent {\bf Takeo Ohsawa (Nagoya)}\\
Extension problems and notions of convexity
\vskip3mm
Extension theorems from complex analytic subsets play important roles in several complex variables. We shall recall how the solvability of the d-bar equation with $L^2$ estimates leads to extension theorems and report recent results on a rigidity question. We shall also discuss generalizations to $q$-convex manifolds, based on my paper published in 2007.


\end{document}